\section{Thread Search}\label{sec:thread_search}

Via the \menu{Search,Thread Search} menu, the appropriate plug-in can be shown or hidden as a tab in the Messages Console. In \codeblocks, a preview can be displayed for the occurrence of a character string in a file, workspace or directory. In doing so, the list of search results will be displayed on the right-hand side of the ThreadSearch Console. By clicking an entry in the list, a preview is displayed on the left-hand side. By double-clicking in the list, the selected file is opened in the \codeblocks editor.

\hint{The scope of file extensions to be included in the search, is preset and might have to be adjusted.}

\subsection{Features}

ThreadSearch plugin offers the following features:

\begin{itemize}
\item Multi-threaded \samp{Search in files}
\item Internal read-only editor to preview the results
\item File open in editors notebook
\item Contextual menu \samp{Find occurrences} to start a search in files with the word under cursor
\end{itemize}

\screenshot[hbt!][width=\columnwidth]{threadsearch_panel}{Thread Search Panel}

\subsection{Usage}

\begin{enumerate}
\item Configure your search preferences (see \pxref{fig:threadsearch_options})

Once the plugin is installed, there are 4 ways to run a search:

\begin{enumerate}
\item Type/Select a word in the search combo box and press enter or click on Search on the Thread search panel of the Messages notebook.
\item Type/Select a word in the toolbar search combo box and press enter or click on Search button.
\item Right click on any \samp{word} in active editor and click on \samp{Find occurrences}.
\item Click on Search/Thread search to find the current word in active editor.
\hint{Items 1, 2 and 3 may not be available according to current configuration.}
\end{enumerate}
\item Click again on the search button to cancel current search.
\item A single click on a result item displays it on the preview editor at right location.
\item A double click on a result item opens or set an editor in editors notebook at right location.
\end{enumerate}

\subsection{Configuration}

To access ThreadSearch plugin configuration panel click on (see \pxref{fig:threadsearch_options}):

\screenshot{threadsearch_options}{Configuration of Thread Search}

\begin{enumerate}
\item Options button on Messages notebook Thread search panel.
\item Options button on Thread search toolbar.
\item Settings/Environment menu item and then on the Thread search item on the left columns.
\end{enumerate}

\hint{Items 1, 2 and 3 may not be available according to current configuration.}

Search in part defines the set of files that will be analysed.

\begin{itemize}
\item Project and Workspace checkboxes are mutually exclusive.
\item Directory path can be edited or set with Select button.
\item Mask is the set a file specifications separated by \samp{;}. For example: \file{*.cpp;*.c;*.h.}
\end{itemize}

\subsection{Options}

\begin{description}
\item[Whole word] if checked, line matches search expression if search expression is found with no alpha-numeric \codeline{+'_'} before and after.
\item[Start word] if checked, line matches search expression if search expression is found at the beginning of a word, ie no alpha-numeric \codeline{+'_'} before search expression.
\item[Match case] if checked, the search is case sensitive.
\item[Regular expression] the search expression is a regular expression.
\end{description}

\hint{If you want to search for regular expressions like \codeline{\n} you will have to set the option \menu{Use Advanced RegEx searches} via the menu \menu{Settings,Editor,General Settings}.}

\subsection{Thread search options}

\begin{description}
\item[Enable \samp{Find occurrences contextual menu item}] If checked, the Find occurrences of \samp{Focused word} entry is added to the editor contextual menu.
\item[Use default options when running \samp{Find occurrences}] If checked, a set of default options is applied to the searches launched with the \samp{Find occurrences} contextual menu item.Per defaut option \samp{Whole word} and \samp{Match case} is enabled.
\end{description}

\subsection{Layout}

\begin{description}
\item[Display header in log window] if checked, the header are displayed in the results list control.
\hint{If unchecked, the columns are no longer resizeable but space is spared.}
\item[Draw lines between columns] Draws lines between columns in list mode.
\item[Show ThreadSearch toolbar] Display the toolbar of Thread Search plugin.
\item[Show search widgets in ThreadSearch Messages panel] If checked, only the results list control and the preview editor are displayed. All other search widgets are hidden (spares space).
\item[Show code preview editor] Code preview can be hidden either with this check box or with a double click on the splitter window middle border. This is where it can be shown again. 
\end{description}

\subsection{Panel Management}

You can choose different modes how the the ThreadSearch window is managed. With the setting \samp{Message Notebook} the ThreadSearch window will be a dockable window in the message panel. If you choose the setting \samp{Layout} you will be able to undock the window from the message panel and put it somewhere else.

\subsection{Logger Type}

The view of the search results can be displayed in different ways. The setting \samp{List} displays all occurrences as list. The other mode \samp{Tree} gathers all occurrences within a file as a node. 
