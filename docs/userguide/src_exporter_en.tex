\section{Source Code Exporter}\label{sec:src_exporter}

The necessity occurs frequently of transferring source code to other applications or to e-mails. If the text is simply copied, formatting will be lost, thus rendering the text very unclear.
The \codeblocks export function serves as a remedy for such situations. The required format for the export file can be selected via \menu{File,Export}. The program will then adopt the file name and target directory from the opened source file and propose these for saving the export file. The appropriate file extension in each case will be determined by the export format. The following formats are available.

\begin{description}
\item[html] A text-based format which can be displayed in a web browser or in word processing applications.
\item[rtf] The Rich Text format is a text-based format which can be opened in word processing applications such as Word or OpenOffice.
\item[odt] Open Document Text format is a standardised format which was specified by Sun and O'Reilly. This format can be processed by Word, OpenOffice and other word processing applications.
\item[pdf] The Portable Document Format can be opened by applications such as the Acrobat Reader.
\end{description}
