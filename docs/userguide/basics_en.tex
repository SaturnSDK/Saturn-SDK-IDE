\chapter{\codeblocks Project Management}

The instructions for \pxref{sec:variables_types} and \pxref{sec:build_codeblocks} are official documentations of the \codeblocks Wiki site and available in english only.

The below illustration shows the design of the \codeblocks user interface.

\figures[H][width=\columnwidth]{codeblocks}{IDE \codeblocks}

\begin{description}
\item[Management] This window contains the interface \menu{Projects} which will in the following text be referred to as the project view. This view show all the projects opened in \codeblocks at a certain time. The \samp{Symbols} tab of the Management window shows symbols, variables etc..
\item[Editor] In the above illustration, a source named \file{hello.c} is opened with syntax highlighting in the editor.
\item[Open files list] shows a list of all files opened in the editor, in this example: \file{hello.c}.
\item[CodeSnippets] can be displayed via the menu \menu{View,CodeSnippets}. Here you can manage text modules, links to files and links to urls.
\item[Logs \& others]. This window is used for outputting search results, log messages of a compiler etc..
\end{description}

The status bar gives an overview of the following settings:

\begin{itemize}
\item Absolute path of an opened file in the editor.
\item The editor uses the default character encoding of your host operating system. This setting will be displayed with \codeline{default}.
\item Row and column number of the current cursor position in the editor.
\item The configured keyboard mode for inserting text (Insert or Overwrite).
\item Current state of a file. A modified file will be marked with \codeline{Modified} otherwise this entry is empty.
\item The permission of a file. A file with read only settings will display \codeline{Read only} in the status bar. In the window \menu{Open files list} these files will be emphasised with a lock as icon overlay.
\item If you start \codeblocks with the command line option \opt{--personality=\var{profile}} then the status bar will show the currently used profile, otherwise \codeline{default} will be shown. The settings of \codeblocks are stored in the corresponding configuration file \file{\var{personality}.conf}.
\end{itemize}

\codeblocks offers a very flexible and comprehensive project management. The following text will address only some of the features of the project management.

\section{Project View}\label{sec:categories}

In \codeblocks, the sources and the settings for the build process are stored in a project file \file{\var{name}.cbp}. C/C++ sources and the corresponding header files are the typical components of a project. The easiest way to create a new project is  executing the command \menu{File,Project} and selecting a wizard. Then you can add files to the project via the context menu \menu{Add files} in the Management window. \codeblocks governs the project files in categories according to their file extensions. These are the preset categories:

\begin{description}
\item[Sources] includes source files with the extensions \file{*.c;*.cpp;}.
\item[ASM Sources] includes source files with the extensions \file{*.s;*.S;*.ss;*.asm}.
\item[Headers] includes, among others, files with the extension \file{*.h;}.
\item[Resources] includes files for layout descriptions for wxWidgets windows with the extensions \file{*.res;*.xrc;}. These file types are shown in the \samp{Resources} tab of the Manangement window.
\end{description}

The settings for types and categories of files can be adjusted via the context menu \menu{Project tree,Edit file types \& categories}. Here you can also define custom categories for file extensions of your own. For example, if you wish to list linker scripts with the \file{*.ld} extension in a category called \file{Linkerscript}, you only have to create the new category.

\hint{If you deactivate \menu{Project tree,Categorize by file types} in the context menu, the category display will be switched off, and the files will be listed as they are stored in the file system.}

\section{Notes for Projects}

In \codeblocks, so-called notes can be stored for a project. These notes should contain short descriptions or hints for the corresponding project. By displaying this information during the opening of a project, other users are provided with a quick survey of the project. The display of notes can be switched on or off in the Notes tab of the Properties of a project.

\section{Project Templates}

\codeblocks is supplied with a variety of project templates which are displayed when creating a new project. However, it is also possible to store custom templates for collecting your own specifications for compiler switches, the optimisation to be used, machine-specific switches etc. in templates. These templates will be stored in the \file{Documents and Settings\osp \var{user}\osp Application Data\osp codeblocks\osp UserTemplates} directory. If the templates are to be open to all users, they have to be copied to a corresponding directory of the \codeblocks installation. These templates will then be displayed at the next startup of \codeblocks under \menu{New, Project,User templates}.

\hint{The available templates in the Project Wizard can be edited by selection via right-click.}

\section{Create Projects from Build Targets}

In projects it is necessary to have different variants of the project available. Variants are called Build Targets. They differ with respect to their compiler options, debug information and/or choice of files. A Build Target can also be outsourced to a separate project. To do so, click \menu{Project,Properties}, select the variant from the tab \samp{Build Targets} and click the \samp{Create project from target} button (see \pxref{fig:build_targets}).

\screenshot{build_targets}{Build Targets}

\section{Virtual Targets}

Projects can be further structured in \codeblocks by so-called Virtual Targets. A frequently used project structure consists of two Build Targets, one \samp{Debug} Target which contains debug information and one \samp{Release} Target without this information. By adding Virtual Targets via \menu{Project,Properties,Build Targets} individual Build Targets can be combined. For example, a Virtual Target \samp{All} can create the Targets Debug and Release simultaneously. Virtual Targets are shown in the symbol bar of the compiler under Build Targets.

\section{Pre- and Postbuild steps}\label{sec:pre_postbuild}

\codeblocks makes it possible to perform additional operations before or after compiling a project. These operations are called Prebuilt or Postbuilt Steps. Typical Postbuilt Steps are:

\begin{itemize}
\item Creating an Intel Hexformat from a finished object
\item Manipulating objects by \cmdline{objcopy}
\item Generating dump files by \cmdline{objdump}
\end{itemize}

\genterm{Example}

Creating a Disassembly from an object under Windows. Piping to a file requires calling \cmdline{cmd} with the \opt{/c} option.

\begin{cmd}
cmd /c objdump -D name.elf > name.dis
\end{cmd}

Archiving a project can be another example for a Postbuilt Step. For this purpose, create a Build Target \samp{Archive} and include the following instruction in the Postbuilt Step:

\begin{cmd}
zip -j9 $(PROJECT_NAME)_$(TODAY).zip src h obj $(PROJECT_NAME).cbp
\end{cmd}

With this command, the active project and its sources, header and objects will be packed as a zip file. In doing so, the Built-in variables \codeline{$(PROJECT_NAME)} and \codeline{$(TODAY)}, the project name and the current date will be extracted (see \pxref{sec:builtin_variables}). After the execution of the Target \samp{Archive}, the packed file will be stored in the project directory.

In the \file{share/codeblocks/scripts} directory you will find some examples for scripts. You can add a script via menu \menu{Settings,Scripting} and register in a menu. If you execute e.g. the script \file{make\_dist} from the menu then all files belonging to a project will be compressed in an archive \file{\var{project}.tar.gz}.

\section{Adding Scripts in Build Targets}

\codeblocks offers the possibility of using menu actions in scripts. The script represents another degree of freedom for controlling the generation of your project.

\hint{A script can also be included at a Build Target.}

\section{Workspace and Project Dependencies}

In \codeblocks, multiple projects can be open. By saving open projects via \menu{File,Save workspace} you can collect them in a single workspace under \file{\var{name}.workspace}. If you open \file{\var{name}.workspace} during the next startup of von \codeblocks, all projects will show up again.

Complex software systems consist of components which are managed in different \codeblocks projects. Furthermore, with the generation of such software systems, there are often dependencies between these projects.

\genterm{Example}

A project A contains fundamental functions which are made available to other projects in the form of a library. Now, if the sources of this project are modified, then the library has to be rebuilt. To maintain consistency between a project B which uses the functions and project A which implements the functions, project B has to depend on project A. The necessary information on the dependencies of projects is stored in the relevant workspace, so that each project can be created separately. The usage of dependencies makes it also possible to control the order in which the projects will be generated. The dependencies for projects can be set via the selecting the menu \menu{Project,Properties} and then clicking the \samp{Project's dependencies} button.

\section{Including Assembler files}

In the Management window of the Project View, Assembler files are shown in the \file{ASM Sources} category. The user can change the listing of files in categories (see \pxref{sec:categories}). Right-clicking one of the listed Assembler files will open a context menu. Select \menu{Properties} to open a new window. Now select the \samp{Build} tab and activate the two fields \samp{Compile file} and \samp{Link file}. Then select the \samp{Advanced} tab and execute the following steps:

\begin{enumerate}
\item Set \samp{Compiler variable} to CC
\item Select the compiler under \samp{For this compiler}
\item Select \samp{Use custom command to build this file}
\item In the window, enter:
\begin{code}
$compiler $options $includes <asopts> -c $file -o $object
\end{code}
\end{enumerate}

The \codeblocks variables are marked by \codeline{$} (see \pxref{sec:command_macros}). They are set automatically so that you only have to replace the Assembler option \var{asopt} by your own settings.

\section{Editor and Tools}

\subsection{Default Code}

The company's Coding Rules require source files to have a standard design. \codeblocks makes it possible to include a predefined content at the beginning of a file automatically when creating new C/C++ sources and headers. This predefined content is called default code. This setting can be selected under \menu{Stettings,Editor} Default Code. A new file can be created via the menu \menu{File,New,File}.

\genterm{Example}

\begin{code}
/*************************************************************************
 *  Project:
 *  Function:
 *************************************************************************
 *  $Author: mario $
 *  $Name:  $
 *************************************************************************
 *
 *  Copyright 2007 by company name
 *
 ************************************************************************/
\end{code}

\subsection{Abbreviation}\label{sec:abbreviation}

A lot of typing can be saved in \codeblocks by defining abbreviation. This is done by selecting \menu{Settings,Editor} and defining the abbreviations under the name \var{name}, which can then be called by the keyboard shortcut Ctrl-J (see \pxref{fig:abbreviation}).

\screenshot{abbreviation}{Defining abbreviations}

Parametrisation is also possible by including variables \codeline{$(NAME}) in the abbreviations.

\begin{code}
#ifndef $(Guard token)
#define $(Guard token)
#endif // $(Guard token)
\end{code}

When performing the abbreviation \var{name} in the source text and performing Ctrl-J, the content of the variable is requested and included.
%Inherit Class
%Im Editor kann durch Auswahl von Inherit Class über die rechte Maustaste. ???

\subsection{Personalities}\label{sec:personalities}

\codeblocks settings are saved as application data in a file called \file{\var{user}.conf} in the \file{codeblocks} directory. This configuration file contains information such as the last opened projects, settings for the editor, display of symbol bars etc. By default, the \samp{default} personality is set so that the configuration is stored in the file \file{default.conf}. If \codeblocks is called from the command line with the parameter \cmdline{--personality=myuser}, the settings will be stored in the file \file{myuser.conf}. If the profile does not exist already, it will automatically be created. This procedure makes it possible to create the corresponding profiles for different work steps. If you start \codeblocks from the command line with the additional parameter\cmdline{--personality=ask}, a selection box will be displayed for all the available profiles.

\hint{The name of the current profile/personality is displayed in the right corner of the status bar.}

\subsection{Configuration Files}

The \codeblocks settings are stored in the \file{default.conf} profile in the \file{codeblocks} directory of your Application Data. When using personalities (see \pxref{sec:personalities}), the configuration details will be stored in the \file{\var{personality}.conf} file.

The tool \cmdline{cb\_share\_conf}, which can be found in the \codeblocks installation directory, is used for managing and storing these settings.

If you wish to define standard settings for several users of a computer, the configuration file \file{default.conf} has to be stored in the directory \file{\osp Documents and Settings\osp Default User\osp Application Data\osp codeblocks}. During the first startup, \codeblocks will copy the presettings from \samp{Default User} to the application data of the current users.

To create a portable version of \codeblocks on a USB stick, proceed as follows. Copy the \codeblocks installation to a USB stick and store the configuration file \file{default.conf} in this directory. The configuration will be used as a global setting. Please take care that the file is writeable, otherwise changes of the configuration cannot be stored.

\subsection{Navigate and Search}

In \codeblocks there are different ways of quick navigation between files and functions. Setting bookmarks is a typical procedure. Via the shortcut Ctrl-B a bookmark is set or deleted in the source file. Via Alt-PgUp you can jump to the previous bookmark, and via Alt-PgDn you can jump to the next bookmark.

If you select the workspace or a project in the workspace in the project view you will be able to search for a file in the project. Just select \menu{Find file} from the context menu, then type the name of the file and the file will be selected. If you hit return this file will be opened in the editor (see \pxref{fig:project_find_file}).

\screenshot[H][width=.5\columnwidth]{project_find_file}{Searching for files}

In \codeblocks you can easily navigate between header/source files like:

\begin{enumerate}
\item Set cursor at the location where a header file is include and open this file via the context menu \menu{open include file} (see \pxref{fig:open_header})
\item Swap between header and source via the context menu \menu{Swap header/source}
\item Select e.g. a define in the editor and choose \menu{Find declaration} from the context menu to open the file with its declaration.
\end{enumerate}

\screenshot{open_header}{Opening of a header file}

\codeblocks offeres several ways of searching within a file or directory. The dialogue box for searching is opened via \menu{Search,Find} (Ctrl-F) or \menu{Find in Files} (Ctrl-Shift-F).

Alt-G and Ctrl-Alt-G are another useful functions. The dialogue which will open on using this shortcut, lets you select files/functions and then jumps to the implementation of the selected function (see \pxref{fig:select_function}) or opens the selected file in the editor. You may use wildcards like \codeline{*} or \codeline{?} etc. for an incremental search in the dialog.

\screenshot[][width=.5\columnwidth]{select_function}{Search for functions}

\hint{With the Ctrl-PgUp shortcut you can jump to the previous function, and via Ctrl-PgDn you can jump to the next function.}

In the editor, you can switch between the tabs with the open files via Ctrl-Tab. Alternatively you can set \samp{Use Smart Tab-switching scheme} in \menu{Settings,Notebook appearance}, then Ctrl-Tab will bring up an Open Tabs window in which all the open files will be listed which can then be selected by mouse-click (see \pxref{fig:tab_scheme}). You can use the shortcut Ctrl-Tab in the management window to switch between the different tabs.

\screenshot{tab_scheme}{Settings of switching between tabs}

A common procedure when developing software is to struggle with a set of functions which are implemented in different files. The Browse Tracker plugin will help you solve this problem by showing you the order in which the files were selected. You can then comfortably navigate the function calls (see \pxref{sec:browsetracker}).

The display of line numbers in \codeblocks can be activated via \menu{Settings,General Settings} in the field \samp{Show line numbers}. The shortcut Ctrl-G or the menu command \menu{Search,Goto line} will help you jump to the desired line.

\hint{If you hold the Ctrl key and then select text in the \codeblocks editor you can perform e.g. a Google search via the context menu.}

\subsection{Symbol view}

The \codeblocks Management window offers a tree view for symbols of C/C++ sources for navigating via functions or variables. As the scope of this view, you can set the current file or project, or the whole workspace. The following categories exist for the symbols:

\screenshot{symbols}{Symbol view}

\begin{description}
\item[Global functions] Lists the implementation of global functions.
\item[Global typedefs] Lists the use of \codeline{typedef} definitions.
\item[Global variables] Displays the symbols of global variables.
\item[Preprocessor symbols] Lists the pre-processor directives created by \codeline{#define}.
\end{description}

Structures and classes are displayed below the pre-processor symbols. If a category is selected by mouse-click, the found symbols will be displayed in the lower part of the window (see \pxref{fig:symbols}). Double-clicking the symbol will open the file in which the symbol is defined or the function implemented, and jumps to the corresponding line.

\hint{In the editor, a list of the classes can be displayed via the context menus \menu{Insert Class method declaration implementation} or \menu{All class methods without implementation}.}
%split view horizontally, vertically

\subsection{Including external help files}

The \codeblocks development environment supports the inclusion of external help files via the menu \menu{Settings,Environment}. Include the manual of your choice in the chm format in \menu{Help Files} select \samp{this is the default help file} (see \pxref{fig:help_files}). The entry \codeline{$(keyword)} is a placeholder for a select item in your editor. Now you can select a function in an opened source file in \codeblocks by mouse-click, and the corresponding documentation will appear while pressing F1.

If you have included multiple help files, you can select a term in the editor and choose a help file from the context menu \menu{Locate in} for \codeblocks to search in.

\screenshot{help_files}{Settings for help files}

In \codeblocks you can add even support for man pages. Just add a entry \menu{man} and specify the path as follows.

\begin{cmd}
man:/usr/share/man
\end{cmd}

\codeblocks provides an \samp{Embedded HTML Viewer}, which can be used to display simple html file and find keywords within this file. Just configure the path to the html file, which should be parsed and enable the checkbox \menu{Open this file with embedded help viewer} via the menu \menu{Settings,Environment,Help Files}.

\screenshot{embedded_html_viewer}{Embedded HTML Viewer}

\hint{If you select a html file with a double-click within the file explorer (see \pxref{sec:file_explorer}) then the embedded html viewer will be started, as long as no association for html files is made in file extensions handler.}
% \section{Scripting}
%
% \codeblocks in Console Modus + Scripts

\subsection{Including external tools}

Including external tools is possible in \codeblocks via \menu{Tools,Configure Tools,Add}. Built-in variables (see \pxref{sec:builtin_variables}) can also be accessed for tool parameters. Furthermore there are several kinds of launching options for starting external applications. Depending on the option, the externally started applications are stopped when \codeblocks is quit. If the applications are to remain open after quitting \codeblocks, the option \menu{Launch tool visible detached} must be set.

\section{Tips for working with \codeblocks}

In this chapter we will present some useful settings in \codeblocks.

\subsection{Configuring environmental variables}

The configuration for an operating system is specified by so-called environmental variables. The environmental variable \codeline{PATH} for example contains the path to an installed compiler. The operating system will process this environmental variable from beginning to end, i.e. the entries at the end will be searched last. If different versions of a compiler or other applications are installed, the following situations can occur:

\begin{itemize}
\item An incorrect version of a software is called
\item Installed software packages call each other
\end{itemize}

So it might be the case that different versions of a compilers or other tools are mandatory for different projects. One possibility in such a case is to change the environmental variables in the system control for every project. However, this procedure is error-prone and not flexible. For this requirement, \codeblocks offers an elegant solution. Different configurations of environmental variables can be created which are used only internally in \codeblocks. Additionally, you can switch between these configurations. The \pxref{fig:env_variables} shows the dialogue which you can open via \samp{Environment Varibales} under \menu{Settings,Environment}. A configuration is created via the \samp{Create} button.

\screenshot{env_variables}{Environmental variables}

Access and scope of the environmental variables created here, is limited to \codeblocks. You can expand these environmental variables just like other \codeblocks variables via \codeline{$(NAME)}.

\hint{A configuration for the environmental variable for each project can be selected in the context menu \menu{Properties} of the \samp{EnvVars options} tab.}

\genterm{Example}

You can write the used environment into a postbuild Step (see \pxref{sec:pre_postbuild}) in a file \file{\var{project}.env} and archive it within your project.

\begin{cmd}
cmd /c echo \%PATH\%  > project.env
\end{cmd}

or under Linux

\begin{cmd}
echo \$PATH > project.env
\end{cmd}

\subsection{Switching between projects}

If several projects or files are opened at the same time, the user needs a way to switch quickly between the projects or files. \codeblocks has a number of shortcuts for such situations.

\begin{description}
\item[Alt-F5] Activates the previous project from the project view.
\item[Alt-F6] Activates the next project from the project view.
\item[F11] Switches within the editor between a source file \file{\var{name}.cpp} and the corresponding header file \file{\var{name}.h}
\end{description}

\subsection{Extended settings for compilers}

During the build process of a project, the compiler messages are displayed in the Messages window in the Build Log tab. If you wish to receive detailed information, the display can be extended. For this purpose click \menu{Settings,Compiler and Debugger} and select \samp{Other Settings} in the drop-down field.

\screenshot{compiler_debugger}{Setting detail information}

Take care that the correct compiler is selected. The \samp{Full command line} setting in the Compiler Logging field outputs the complete information in the Build Log. In addition, this output can be logged in a HTML file. For this purpose select \samp{Save build log to HTML file when finished}.
Furthermore, \codeblocks offers a progress bar for the build process in the Build Log window which can be activated via the \samp{Display build progress bar} setting.

\subsection{Zooming within the editor}

\codeblocks offers a very efficient editor. This editor allows you to change the size in which the opened text is displayed. If you use a mouse with a wheel, you only need to press the Ctrl key and scroll via the mouse wheel to zoom in and out of the text.

\hint{With the shortcut Ctrl-Numepad-/ or with the menu \menu{Edit,Special commands,Zoom,Reset} the original font size of the active file in the editor is restored.}

\subsection{Block select mode in editor}

\codeblocks supports the block select mode within the editor. Hold the key \samp{ALT} and select a region with the left mouse button and copy or paste your selection. This feature is helpful if you want to select some columns e.g. of an array and copy and paste the content.

\hint{Most Linux window managers use ALT-LeftClickDrag to move a window, so you will have to disable this window manager behavior first for block select to work.}

\subsection{Code folding}

\codeblocks supports so called cold folding. With this feature you can fold e.g. functions within the \codeblocks editor. A folding point is marked by minus symbol in the left margin of
the editor view. In the margin the beginning and the end of a folding point is visible as vertical line. If you click the minus symbol with the left mouse button the code snippet will be folded or unfolded. Via the menu \menu{Edit,Folding} you can select the folding. In the editor you see folded code as continous horizontal line.

\hint{The folding style can be configured via menu \menu{Settings,Editor,Folding}.}

\codeblocks provides the folding feature also for preprocessor directives. To enable this feature select \samp{Fold preprocessor commands} via the menu \menu{Settings,Editor} in the folding entry.

Another possibility is to set user defined folding points. The start of folding point is entered as comment with a opening bracket and the end is market with a comment with a closing bracket.

\begin{code}
//{
code with user defined folding
//}
\end{code}

\subsection{Auto complete}

If you open a open a project in \codeblocks the \samp{Search directories} of your compiler and the project, the sources and headers of your project are parsed. In addition the keyowrds of the corresponding lexer file are parsed. The parse information is used for the auto complete feature in \codeblocks. Please check the settings for the editor if this feature is enabled. The auto completion is accessible with the shortcut Ctrl-Space. Via the menu \menu{Settings,Editor,Syntax highlighting} you can add user defined keywords to your lexer.

\subsection{Including libraries}

In the build options of a project, you can add the used libraries via the \samp{Add} button in the \samp{Link libraries} entry of the \samp{Linker Settings}. In doing so, you can either use the absolute path to the library or just give the name without the \file{lib} prefix and file extension.

\genterm{Example}

For a library called \file{\var{path}\osp libs\osp lib\var{name}.a}, just write \file{\var{name}}. The linker with the corresponding search paths will then include the libraries correctly.

\hint{Another way to include libraries is documented in \pxref{sec:lib_finder}.}

\subsection{Object linking order}

During compiling, objects \file{name.o} are created from the sources \file{name.c/cpp}. The linker then binds the individual objects into an application \file{name.exe} or for the embedded systems \file{name.elf}. In some cases, it might be desirable to predefine the order in which the objects will be linked. In \codeblocks, this can be achieved by assigning priorities. In the context menu \menu{Properties}, you can define the priorities of a file in the Build tab. A low priority will cause the file to be linked earlier.

\subsection{Autosave}

\codeblocks offers ways of automatically storing projects and source files, or of creating backup copies. This feature can be activated in the menu \menu{Settings,Environment,Autosave}. In doing so, \samp{Save to .save file} should be specified as the method for creating the backup copy.

\subsection{Settings for file extensions}\label{sec:file_extension}

In \codeblocks, you can choose between several ways of treating file extensions. The settings dialogue can be opened via \menu{Settings,Files extension handling}.
You can either use the applications assigned by Windows for each file extension (open it with the associated application), or change the setting for each extensions in such a way that either a user-defined program will start (launch an external program), or the file will be opened in the \codeblocks editor (open it inside Code::Blocks editor).

\hint{If a user-defined program is assigned to a certain file extension, the setting \samp{Disable Code::Blocks while the external program is running} should be deactivated because otherwise \codeblocks will be closed whenever a file with this extension is opened.}

\section{\codeblocks at the command line}

IDE \codeblocks can be executed from the command line without a graphic interface. In such a case, there are several switches available for controlling the build process of a project. Since \codeblocks is thus scriptable, the creation of executables can be integrated into your own work processes.

\begin{cmd}
codeblocks.exe /na /nd --no-splash-screen --built <name>.cbp --target='Release'
\end{cmd}

\begin{optentry}
\item[\var{filename}] Specifies the project \file{*.cbp} filename or workspace \file{*.workspace} filename. For instance, \var{filename} may be \file{project.cbp}. Place this argument at the end of the command line, just before the output redirection if there is any.
\item[/h, --help] Shows a help message regarding the command line arguments.
\item[/na, --no-check-associations] Don't perform any file association checks (Windows only).
\item[/nd, --no-dde] Don't start a DDE server (Windows only).
\item[/ns, --no-splash-screen] Hides the splash screen while the application is loading.
\item[/d, --debug-log] Display the debug log of the application.
\item[--prefix=\var{str}] Sets the shared data directory prefix.
\item[/p, --personality=\var{str}, --profile=\var{str}] Sets the personality to use. You can use ask as the parameter to list all available personalities.
\item[--rebuild] Clean and build the project or workspace.
\item[--build] Build the project or workspace.
\item[--target=\var{str}] Sets target for batch build. For example \cmdline{--target='Release'}.
\item[--no-batch-window-close] Keeps the batch log window visible after the batch build is completed.
\item[--batch-build-notify] Shows a message after the batch build is completed.
\item[--safe-mode] All plugins are disabled on startup.
\item[$>$ \var{build log file}] Placed in the very last position of the command line, this may be used to redirect standard output to log file. This is not a codeblock option as such, but just a standard DOS/*nix shell output redirection.
\end{optentry}

\section{Shortcuts}

Even if an IDE such as \codeblocks is mainly handled by mouse, keyboard shortcuts are nevertheless a very helpful way of speeding up and simplifying work processes. In the below table, we have collected some of the available keyboard shortcuts.

\subsection{Editor}

\begin{tabular}{|l|l|}\hline
Function		&	Shortcut Key\\ \hline
Undo last action 	&	Ctrl-Z\\ \hline
Redo last action 	&	Ctrl-Shift-Z\\ \hline
Swap header / source 	&	F11\\ \hline
Comment highlighted code &	Ctrl-Shift-C\\ \hline
Uncomment highlighted code & 	Ctrl-Shift-X\\ \hline
Auto-complete / Abbreviations & 	Ctrl-Space/Ctrl-J\\ \hline
Toggle bookmark 	&	Ctrl-B\\ \hline
Goto previous bookmark 	&	Alt-PgUp\\ \hline
Goto next bookmark 	&	Alt-PgDown\\ \hline
\end{tabular}

This is a list of shortcuts provided by the \codeblocks editor component. These shortcuts cannot be rebound.

\begin{tabular}{|l|l|}\hline
Create or delete a bookmark	&	Ctrl-F2\\ \hline
Go to next bookmark		&	F2\\ \hline
Select to next bookmark		&	Alt-F2\\ \hline
Find selection.			& 	Ctrl-F3\\ \hline
Find selection backwards. 	&	Ctrl-Shift-F3\\ \hline
Find matching preprocessor conditional, skipping nested ones. &	Ctrl-K\\ \hline
\end{tabular}

\subsection{Files}

\begin{tabular}{|l|l|}\hline
Function 		&	Shortcut Key\\ \hline
New file or project 	&	Ctrl-N\\ \hline
Open existing file or project &	Ctrl-O\\ \hline
Save current file 	&	Ctrl-S\\ \hline
Save all files 		&	Ctrl-Shift-S\\ \hline
Close current file 	&	Ctrl-F4/Ctrl-W\\ \hline
Close all files 	&	Ctrl-Shift-F4/Ctrl-Shift-W\\ \hline
\end{tabular}

\subsection{View}

\begin{tabular}{|l|l|}\hline
Function 		&	Shortcut Key\\ \hline
Show / hide Messages pane	&	F2\\ \hline
Show / hide Management pane 	&	Shift-F2\\ \hline
Activate prior (in Project tree) & 	Alt-F5\\ \hline
Activate next (in Project tree)  &	Alt-F6\\ \hline
\end{tabular}

\subsection{Search}

\begin{tabular}{|l|l|}\hline
Function 	&	Shortcut Key\\ \hline
Find 		&	Ctrl-F\\ \hline
Find next 	&	F3\\ \hline
Find previous 	&	Shift-F3\\ \hline
Find in files 	&	Crtl-Shift-F\\ \hline
Replace 	&	Ctrl-R\\ \hline
Replace in files &	Ctrl-Shift-R\\ \hline
Goto line 	&	Ctrl-G\\ \hline
Goto file 	&	Alt-G\\ \hline
Goto function 	&	Ctrl-Alt-G\\ \hline
\end{tabular}

\subsection{Build}

\begin{tabular}{|l|l|}\hline
Function 	&	Shortcut Key\\ \hline
Build 		&	Ctrl-F9\\ \hline
Compile current file	&	Ctrl-Shift-F9\\ \hline
Run		&	Ctrl-F10\\ \hline
Build and Run 	&	F9\\ \hline
Rebuild 	&	Ctrl-F11\\ \hline
\end{tabular}
